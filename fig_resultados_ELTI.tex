% fig_resultados_ELTI.tex
% Figura generada en TikZ para representar resultados de la simulación EEG en epilepsia temporal izquierda (ELTI)
% Autor: Roberto J Romero De Anda
% Licencia: CC BY-NC-ND 4.0

\begin{figure}[H]
\centering
\begin{tikzpicture}[scale=1.0]

% Marco general
\draw[thick] (0,0) rectangle (13,9.5);

% Subgráfico A: Dimensión Fractal
\draw[thick] (0.5,5.5) rectangle (6.3,9);
\node at (3.4,8.8) {\textbf{(A) Dimensión Fractal en Canales ELTI}};
\draw[->] (0.8,6) -- (0.8,8.5);
\draw[->] (0.8,6) -- (6,6);

\node[rotate=90] at (0.4,7.25) {\footnotesize DF};
\node at (3.4,5.7) {\footnotesize Tiempo (s)};

\foreach \i/\x in {F7/1.2, T7/2.2, C3/3.2, P7/4.2} {
    \draw[blue, thick] plot[smooth] coordinates {(\x,6.1) (\x+0.2,7.5) (\x+0.4,6.8) (\x+0.6,7.0)};
    \node[below] at (\x+0.3,6.1) {\scriptsize \i};
}

% Subgráfico B: Señal EEG en T7
\draw[thick] (6.7,5.5) rectangle (12.5,9);
\node at (9.6,8.8) {\textbf{(B) Señal EEG Simulada en T7}};
\draw[->] (7,6) -- (7,8.5);
\draw[->] (7,6) -- (12.2,6);

\node[rotate=90] at (6.6,7.25) {\footnotesize Voltaje ($\mu$V)};
\node at (9.6,5.7) {\footnotesize Tiempo (s)};

\draw[red, thick, samples=60, domain=7:12.2] plot (\x, {7 + 0.8*sin(2*pi*3*\x r) + 0.4*sin(2*pi*15*\x r)});

% Etiquetas generales
\node at (6.5,-0.5) {\textit{Figura: Resultados Simulados en Modelo TCBC para Epilepsia ELTI}};

\end{tikzpicture}
\caption{
    Resultados de simulación MATLAB para epilepsia temporal izquierda (ELTI): 
    (A) Dimensión fractal reducida en canales F7, T7, C3, P7 bajo umbral patológico (<1.38), 
    (B) Señal EEG simulada en canal T7 con oscilaciones theta y ondas agudas.
}
\label{fig:tcbc_results}
\end{figure}
